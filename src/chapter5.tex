\section{Économie internationale et globalisation} % (fold)
\label{sec:economie_internationale_et_globalisation}

Il est important de savoir que depuis la fin de la seconde guerre mondiale,
les échanges internationaux ainsi que la circulation des capitaux se
sont beaucoup développés.
Les évolutions des taux de change ont des effets sur la compétitivité
des nations et donc sur l'économie réelle.
Cette mondialisation est possible notamment grâce à la diminution
des coûts des transports (feroviaire et maritime).
On s'attechera dans cette section à décrire deux points
\begin{enumerate}
  \item Les théories du \emph{commerce international} qui étudient les flux réels de biens 
  et de services d'une part,
  \item et les théories de la \emph{finance internationale} qui analysent les flux 
  financiers d'autre part.
\end{enumerate}
On discutera également les impacts, qu'ils soient positifs ou négatifs,
de ces théories sur la mondialisation.

\subsection{Commerce international} % (fold)
\label{sub:commerce_international}
Il paraît interessant de noter qu'en moyenne, les petits pays font plus 
de commerce avec les autres pays, que les grands.
Ceci s'expliquent par une production interne moins diversifiée
et à moins grande échelle.


% subsection commerce_international (end)

\subsection{Finance internationale} % (fold)
\label{sub:finance_internationale}

% subsection finance_internationale (end)

% section economie_internationale_et_globalisation (end)