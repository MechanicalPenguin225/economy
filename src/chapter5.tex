\section{Économie internationale et globalisation} % (fold)
\label{sec:economie_internationale_et_globalisation}

Il est important de savoir que depuis la fin de la seconde guerre mondiale,
les échanges internationaux ainsi que la circulation des capitaux se
sont beaucoup développés.
Les évolutions des taux de change ont des effets sur la compétitivité
des nations et donc sur l'économie réelle.
Cette mondialisation est possible notamment grâce à la diminution
des coûts des transports (feroviaire et maritime)
ainsi que la chûte régulière des droits de douane.
On s'attechera dans cette section à décrire deux points
\begin{enumerate}
  \item Les théories du \emph{commerce international} qui étudient les flux réels de biens 
  et de services d'une part,
  \item et les théories de la \emph{finance internationale} qui analysent les flux 
  financiers d'autre part.
\end{enumerate}
On discutera également les impacts, qu'ils soient positifs ou négatifs,
de ces théories sur la mondialisation.

\subsection{Commerce international} % (fold)
\label{sub:commerce_international}
Il paraît interessant de noter qu'en moyenne, les petits pays font plus 
de commerce avec les autres pays, que les grands.
Ceci s'expliquent par une production interne moins diversifiée
et à moins grande échelle.

À titre d'exemple, on notera que les accords au sein de l'Union Européenne
sur le libre-échange ont permis de porter le commerce entre pays européens au tiers
du commerce mondial.


\subsubsection{Régulation du commerce mondial} % (fold)
\label{ssub:regulation_du_commerce_mondial}
On va maintenant brièvement décrire les différents types d'accords internationaux
qui nous ont amenés à la situation actuelle.

Les partenaires principaux du commerce international sont contraints par le
\textsc{gatt} (General Agreements on Tarifs and Trade) par trois principes,
à savoir
\begin{enumerate}
  \item Le principe de \emph{non-discrimination} dicte que tout avantage tarifaire accordé
  à un membre doit être étendu à l'ensemble des membres,
  \item le principe de \emph{réciprocité} dicte qu'on ne peut bénéficier des concessions
  de ses partenaires sans en accorder soi-même,
  \item et le principe de \emph{transparence} qui dicte que les barrières commerciales de
  types quotas doivent être converties en droits de douane.
\end{enumerate}
% subsubsection regulation_du_commerce_mondial (end)
On retrouve ensuite deux \textbf{interdictions} qui viennent compléter ces trois principes
\begin{enumerate}
  \item interdiction de \emph{dumping}, qui correspond à  exporter une marchandise 
  à un prix inférieur à celui pratiqué dans le pays d'origine,
  \item et l'interdiction des \emph{subventions} lorsqu'elles maintiennent des prix
  artificiellement faibles.
\end{enumerate}

% subsection commerce_international (end)

\subsubsection{Théories du commerce international} % (fold)
\label{ssub:theories_du_commerce_international}
% TODO expliquer ce que sont les deux avantages
On retrouve deux types d'avantages internationaux.
\begin{enumerate}
  \item Les \emph{avantages absolus de Smith} est obtenu dans l'échange
  international par la nation qui produit et vend un bien à un prix inférieur
  à celui des nations concurrentes.
  Chaque pays a donc intérêt à se spécialiser dans les productions pour lesquelles
  il détient l'avantage absolu par rapport aux autres nations,
  et à se procurer au moindre côut les productions pour lesquelles 
  il ne possède aucun avantage par rapport à l'extérieur.
  \item Les \emph{avantages comparatifs de Ricardo} stipule que, chaque nation,
  lorsqu'elle se spécialise dans la production pour laquelle elle dispose de la 
  productivité la plus forte ou la moins faible, comparativement à ses partenaires,
  accroît sa richesse nationale. Les richesses mondiales sont donc accrues 
  et il est possible d'avoir des échanges commerciaux bénéfiques même pour 
  une économie qui n'a pas d'avantage absolu.
\end{enumerate}
Cependant, ces deux-ci ne permettent pas d'expliquer tous les phénomènes
actuels observés.
Par exemple, lorsqu'on s'intéresse à l'existence de rendements d'échelle
croissant qui entraînent la concentration de la production d'un bien 
dans un seul pays, il faut faire appel au modèle de \emph{concurrence imparfaite}.

% subsubsection theories_du_commerce_international (end)

\subsection{Finance internationale} % (fold)
\label{sub:finance_internationale}
Avant de détailler comment sont enregistrés toutes les transactions interpays,
on va expliquer en quoi consiste le taux de change et la monnaie de référence mondiale
puisque ceux-ci ont une influence importante sur la finance internationale.

\subsubsection{Taux de change} % (fold)
\label{ssub:taux_de_change}
Le change est globalement déterminé par la confrontation de l'offre et
de la demande d'une monnaie sur le marché.
Le \emph{taux de change nominal} est le nombre d'unités monétaires que l'on 
peut obtenir en échange d'une unité de l'autre monnaie.
Il existe deux type de cotations pour le taux de change.
\begin{enumerate}
  \item la cotation à l'\emph{incertain}, lorsque la monnaie étrangère 
  est exprimée en monnaie nationale, %(ex. 1 = 0.7\euro)
  \item et la cotation au \emph{certain}, lorsque la monnaie nationale est exprimée
  en unité de monnaie étrangère. %(ex. 1\euro = 1.4\dollar)
\end{enumerate}
Pour prendre en compte les différences de pouvoir d'achat qui peuvent exister entre les pays,
on utilise le \emph{taux de change réel}.
Dans le cas du taux de change réel à l'incertain, on a
\[
  e_r = e_n \, \frac{p_e}{p}
\]
avec $e_n$ le taux de change nominal à l'incertain, $p_e$ l'indice des prix étrangers
et $p$ celui des prix dans la zone euro.
Lorsqu'on veut prendre en compte le taux de change \emph{moyen} d'une monnaie
avec un ensemble de devises, on utilisera le \emph{taux de change effectif}
(en faisant une \emph{moyenne pondérée} des différents taux de change bilatéraux).

% subsubsection taux_de_change (end)

\subsubsection{Balance des paiements} % (fold)
\label{ssub:balance_des_paiements}
L'ensemble des échanges de biens et services, mais aussi des échanges financiers,
sont enrégistrés dans la \emph{balance des paiements}.
C'est un document comptable suivant les principes de la \emph{comptabilité en partie double}
qui permet de connaître l'ampleur des échanges et des déséquilibres économiques d'un pays.
C'est donc un compte de \emph{flux} (et non de stocks) tel que le flux d'un résident vers 
un non-résident s'inscrit en \emph{crédit} et le flux inverse en \emph{débit}.

Il arrive dans certains cas que la balance soit \emph{déséquilibrée} (lorsque le déséquilibre 
du solde courant n'est pas exactement compensé par les mouvements de capitaux 
entre résidents et non-résidents).
Dans ce cas, le pays doit avoir recours à un financement monétaire de ces déséquilibres.
On retrouve deux situation de déséquilibre (pour comprendre comment ces déséquilibres
s'inscrivent dans le \textbf{modèle IS-LM-BP},
on se réferera aux \textbf{dernières pages du chapitre~5 du polycopié}).
\begin{enumerate}
  \item La balance des paiements est \emph{excédentaire} et le pays accumule des devises.
  Pour équilibrer celle-ci il s'agira que la balance des capitaux devienne déficitaire à
  travers une \emph{sortie de capitaux}, 
  obtenue elle-même grâce à une \emph{baisse du taux d'intérêt}.
  \item La balance des paiements est \emph{déficitaire} lorsqu'il y a une réduction
  de la masse monétaire du pays à la suite de la baisse de réserves utilisées 
  pour couvrir le déficit de la balance globale.
  Pour équilibrer celle-ci il s'agira que la balance des capitaux devienne excédentaire à
  travers un \emph{afflux de capitaux}, 
  obtenue elle-même grâce à une \emph{hausse du taux d'intérêt}.
\end{enumerate}

% subsubsection balance_des_paiements (end)


% subsection finance_internationale (end)

% section economie_internationale_et_globalisation (end)