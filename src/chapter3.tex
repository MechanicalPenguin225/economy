\part{Financement de l'économie et rôle de la monnaie} % (fold)
\label{prt:financement_de_l_economie_et_role_de_la_monnaie}

À quoit sert la monnaie ? Plusieurs questions se posent, ce chapitre a pour but d'apporter des éléments de réponse. 

La première question que l'on se pose est : qu'est-ce qu'un besoin de financement ? 
Un besoin de financement reprèsente le fait qu'un agent ne puisse financer ses besoins avec ses ressources. Il recourt alors à un \emph{système de crédit} ou un 
\emph{système de financement} (distinction faite par Hicks en 1974).

On note que les ménages et entreprises possédent une capacité de financement mais elle peut s'avérer insuffisante. Un point à analyser est le fait qu'à partir 
des années 80, l'état adopte un position favorable à l'endettement (actuellemet l'endettement reprèsente $90\%$ du PIB).

À l'échelle microéconomique,  ce sont les banques qui finances les différents agents. On favorise ainsi la circulation de la monnaie, de part une promesse de
remboursement futur. Cependant on encourt le risque d'\emph{inflation}. Ce risque s'équilibre par l'augmentation de la demande globale et donc de la croissance 
(augmentation revenus..), éléments qui permettent de rembourser les différents emprunts. Mais si les revenus stagnent, on peut entrer dans une imposibilité de remboursement qui entraine un nouvel emprunt, en arrivant finalement à un surendettement.

Et les marchés financiers ? Ils reprèsentent la deuxième source de financement. Ils permettent d'obtenir différents capitaux, sans passer par les banques. On 
est cependant à la merci des différents risques associés aux mñethodes finacières (spéculation..). Ces marchés peuvant financer des projets sans aucune finalité d'investissement productif.. On crée ainsi des \emph{bulles spéculatives}.

La crise des \emph{subprimes} est un exemple de bulle spéculative. Elle concernait les ménages americains insolvables qui s'étaient endettés poura acheter des
biens immobiliers (avec des crédits dits à prêt hypothécaires, basés sur la veleur du logement). La chute du prix de l'immobilier et la hausse du taux d'térêt a 
provoqué la crise. Mais comment en est-on arrivé là ? On créa des titres à partir de dettes hypothécaires, les banques un prêt hypothécaire, puis réunissaient 
certaines hypothèques en créant des titres "atractif" qui entraient sur les marchés. Les banques ne supportaient aucun risque, et réalisaient des prêts 
insolvables. Cependant l'insolvabilité d'une grande partie des ménages a provoqué, la chute de beaucoup de banques (il est important de comprendre que les 
banques sont aussi des acteurs financiers..). Cet effet provoca un endurcissement des prets qui a ralenti la création de valeurs et la croissance de 
l'économie. Plusieurs plans de rachat par l'état des créances douteuses se sont mit en place. Un image importante de cette crise est la globalité de 
l'économie, la crise americaine à fait trembler l'économie mondiale. 

Quelles mesures aujourd'hui ? Plusieurs institutions (Conseil européen du risuqe financier, Plan de refonte de la régulation de la finance américaine) se sont
 instaurés pour contrôler les actions des marchés financiers. De plus au niveau international le FMI évolue pour répondre aux enjeux de la crise. 
 
Pour analyser les enjeux économiques actuels il nous faut présenter certaines notions.

\begin{tcolorbox}[title=Les agrégats monétaires]
	Ce sont des regroupements d'ensembles homogènes d'actifs monétaires ou non. On les classe ensuite par ordre de liquidité décroissante. 
	\begin{itemize}[label=\ding{69}]
		\item M1: monnaie fiduciare (billets) et les dépôts à vue.
		\item M2: somme de M1 et des depôts moins liquides, par exemple dépôts qui peuvent liquides après un certain terme (Livret A..)
		\item M3: M1+ M2 auxquels on ajoute les exigibilités négociables des institutions financières.
	\end{itemize}
	
\end{tcolorbox}

\section{Le passage de l'économie d'endettement à celle de marché financier} % (fold)
\label{sec:le_passage_de_l_economie_d_endettement_a_celle_de_marche_financier}



% section le_passage_de_l_economie_d_endettement_a_celle_de_marche_financier (end)



% part financement_de_l_economie_et_role_de_la_monnaie (end)

