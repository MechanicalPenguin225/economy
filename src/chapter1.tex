\section{Introduction générale} % (fold)
\label{prt:introduction_generale}

\subsection{L'économie est-elle une science} % (fold)
\label{sec:l_economie_est_elle_une_science}
Il existe 2 raisons principales pouvant nous faire douter de la scientificité de l'économie:

\begin{itemize}[label=\ding{69}]
	\item Avancée progressive sans jamais pouvoir parler de découvertes	
	\item Non respect du critère de scientificité de Popper : \emph{une prosposition est scientifique lorsqu'elle peut être réfutée par l'observation}. Cependant elle n'est pas la seule, la météo par exemple ne le respecte pas. L'économétrie essaie de combler cette lacune.
\end{itemize}

La démarche hypothético-déductive est de plus très criticable, développement de théories et hypothèses sans que l'observation et la vérification ne soient jamais réalisées. Les mathématiques peuvent isoler les économistes de la réalité.

% subsection l_economie_est_elle_une_science (end)

\subsection{Qu'est-ce que l'analyse économique} % (fold)
\label{sec:qu_est_ce_que_l_analyse_economique}

Un constat en économie stipule que les besoins des consommateurs sont illimités et les ressources elles sont limitées. Le consommateur doit donc faire des choix dans l'allocation de ses ressources, c'est ce que l'on appelle le coût d'opportunité. Selon Samuelson la science économique a pour objectif de résoudre les problèmes d'allocations des ressources en répondant à trois questions : \emph{Quoi produire ? Comment produire ? Pour qui produire ?}. Cette définition est cependant trop générale, il nous faut être plus précis. 

\paragraph{Économie positive contre économie normative} % (fold)
\label{par:economie_positive_contre_economie_normative}

\begin{itemize}[label= \ding{69}]
	\item \textbf{Économie positive}: fournir des explications scientifiques et objectives au fonctionnement de l'économie. \emph{Ex: Si on augmente la taxe sur l'essence alors la consommation d'essence diminuera, toutes choses égales par ailleurs}
	\item \textbf{Économie normative}: fournir des recommandations fondées sur des jugements à valeur personnelle.
	 \emph{Ex: L'état doit toujours adopter des mesures de relance économique pour créer des emplois}
\end{itemize}
% paragraph economie_positive_contre_economie_normative (end)

\paragraph{Macroéconomie vs microéconomie} % (fold)
\label{par:macroeconomie_vs_microeconomie}

\begin{itemize}[label= \ding{69}]
	\item \textbf{Macroéconomie}: s'intéresse à l'économie dans sa globalité en analysant des indicateurs globaux (PIB..) et leur relation. Idée \emph{holistique}, comportements individuels s'incrivent dans un contexte global prédeterminé.
	\item  \textbf{Microéconomie:} s'intéresse à chaque groupe d'agents économiques pris individuellements. Idée \emph{individualiste}, étudier les phénomènes en pensant qu'ils peuvent s'expliquer de par les comportements individuels.
\end{itemize}

% paragraph macroeconomie_vs_microeconomie (end)

\paragraph{Analyse conjoncturelle et analyse structurelle} % (fold)
\label{par:analyse_conjoncturelle_et_analyse_structurelle}

\begin{itemize}[label= \ding{69}]
	\item \textbf{Analyse conjoncturelle : }analyse vouée à avoir un effet sur le court terme.
	\item \textbf{Analyse structurelle :} analyse vouée à avoir un effet sur le long terme.
\end{itemize}

% paragraph analyse_conjoncturelle_et_analyse_structurelle (end)

% subsection qu_est_ce_que_l_analyse_economique (end)

\subsection{Évolution dans la conception de l'analyse économique} % (fold)
\label{sec:evolution_dans_la_conception_de_l_analyse_economique}

\begin{itemize}[label= \ding{69}]
	\item \textbf{Physiocrates, XVIII:} production sous l'angle particulier de l'étude de la terre. 
	\item \textbf{Mercantiles, XVI:} étudient les bienfaits économiques du commerce international entre la vieille Europe et le nouveau monde.
	\item \textbf{Smith (1776), Ricardo (1817), Say (Loi de l'offre, 1776-1870):} école de pensée classique, développement de la théorie de la valeur travail, on mesure la valeur par la quantité de travail.
	\item \textbf{Marxistes et socialistes 1870-1920) :} économie politique, vision plus collective et normative de l'économie.
	\item \textbf{Néoclassiques-Marginalistes (1870-1920) :} la valeur du bien provient de l'utilité que l'on tire de sa consomation.
	\item \textbf{Critique Keynésienne (1930):} analyse macro-économique, preuve de stabilité de suproduction et et sous-emploi qui remet en cause les théories d'ajustmeent simulatanés.
	\item \textbf{Hicks, Samuelson (1940-1970):} mélange pensée néoclassique et keynésienne.
	\item \textbf{Monétaristes, Friedman :} injection de liquidités entraine hausse des prix et les agents augmentent donc alors leurs épargnes. Les relances budgétaires ont donc un faible effet à court terme mais impliquent une grosse inflation.
	\item \textbf{Anticipations rationelles, Lucas :} si les agents connaissent le modèle ils peuvent anticiper ses mouvements. Ils peuvent anticiper une hausse des impots suite à un déficit budgétaire et augmentent ainsi leur épargne en annulant l'effet de relance.
	\item \textbf{Nouveaux keynésiens (actuellement) :} les fluctuations son le reflet de l'échec du marché à grande échelle. L'État est nécessaire pour aler en contre des défaillances du marché.
	\item \textbf{Nouveaux classiques (actuellement):} les individus sont rationnels et les marchés sont toujours en équilibre, les fluctuations sont des réponses naturelles et efficaces de l'économie. L'État est inutile, voir nuisible aux fluctuations.
\end{itemize}

% subsection evolution_dans_la_conception_de_l_analyse_economique (end)

% part introduction_generale (end)

